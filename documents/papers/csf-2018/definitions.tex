\newcommand{\ottifc}{Ott-IFC}
\newcommand{\config}[3]{\ensuremath{\langle #1, #2, #3 \rangle}}
\newcommand{\emits}[3]{\ensuremath{#2\downarrow_{#1}#3}}
\newcommand{\proj}[2]{\ensuremath{{#1 \! \upharpoonright  \! {#2}}}}
\newcommand{\step}{{\ensuremath{\Downarrow}}}
\newcommand{\levelOfChan}[1]{\ensuremath {\mathit{labelOfChan}(#1)}}

\newtheorem{definition}{Definition}
\newtheorem{theorem}{Theorem}

%the 3 following lines makes the text in \lstinline the right size. Without it, the text is too small
\makeatletter
\lst@AddToHook{TextStyle}{\let\lst@basicstyle\ttfamily\normalsize\fontfamily{pcr}\selectfont}
\makeatother

\definecolor{grey}{rgb}{0.8,0.8,0.8}
\definecolor{code-background}{RGB}{255, 248, 220}
\definecolor{code-comment}{RGB}{196, 42, 42}
\definecolor{code-linenumber}{rgb}{0.5,0.5,0.5}
\definecolor{code-keyword}{RGB}{148, 0, 211}
\definecolor{coqatoo-pink}      {RGB}{255, 107, 104}
\definecolor{coqatoo-gray}      {RGB}{64, 64, 64}
\definecolor{coqatoo-yellow}    {RGB}{239, 223, 0}
\lstset{
   backgroundcolor=\color{code-background},   	% choose the background color; you must add \usepackage{color} or \usepackage{xcolor}
   basicstyle=\tt\footnotesize,       			% the size of the fonts that are used for the code
   breakatwhitespace=false,         			% sets if automatic breaks should only happen at whitespace
   breaklines=true,                 			% sets automatic line breaking
   captionpos=none,                    			% sets the caption-position to bottom
   commentstyle=\color{code-comment},   		% comment style
   deletekeywords={...},            			% if you want to delete keywords from the given language
%   escapeinside={\%*}{*)},          			% if you want to add LaTeX within your code
   extendedchars=true,              			% lets you use non-ASCII characters; for 8-bits encodings only, does not work with UTF-8
   frame=tb,
   framerule=0pt,
   framextopmargin=0pt,
   framexbottommargin=0pt,
   keepspaces=true,                 			% keeps spaces in text, useful for keeping indentation of code (possibly needs columns=flexible)
   keywordstyle=\color{code-keyword},       						% keyword style
   language=ML,                 				% the language of the code
   morekeywords={Lemma, Proof, Qed, if, then, else, end, skip, stop, read, from, write, to},            % if you want to add more keywords to the set
   morecomment=[l]{\%},
   numbers=none,                    			% where to put the line-numbers; possible values are (none, left, right)
   numbersep=5pt,	                   			% how far the line-numbers are from the code
   numberstyle=\tiny\color{code-linenumber}, 	% the style that is used for the line-numbers
   rulecolor=\color{black}, 	        		% if not set, the frame-color may be changed on line-breaks within not-black text (e.g. comments (green here))
   showspaces=false,            	    		% show spaces everywhere adding particular underscores; it overrides 'showstringspaces'
   showstringspaces=false,          			% underline spaces within strings only
   showtabs=false,                  			% show tabs within strings adding particular underscores
   stepnumber=1,                    			% the step between two line-numbers. If it's 1, each line will be numbered
   stringstyle=\color{black},            		% string literal style
   tabsize=1,                       			% sets default tabsize to 2 spaces
   gobble=0,									% number of characters to remove at the beginning of each line
   mathescape=true,								% to render math symbols in the listing (between $)
   title=\lstname,                   			% show the filename of files included with \lstinputlisting; also try caption instead of title
   belowcaptionskip = 0cm
}