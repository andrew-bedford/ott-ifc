%% For double-blind review submission, w/o CCS and ACM Reference (max submission space)
\documentclass[sigplan,10pt]{acmart}\settopmatter{printfolios=true,printccs=false,printacmref=false}
  
%% For double-blind review submission, w/ CCS and ACM Reference
%\documentclass[sigplan,review,anonymous]{acmart}\settopmatter{printfolios=true}
%% For single-blind review submission, w/o CCS and ACM Reference (max submission space)
%\documentclass[sigplan,review]{acmart}\settopmatter{printfolios=true,printccs=false,printacmref=false}
%% For single-blind review submission, w/ CCS and ACM Reference
%\documentclass[sigplan,review]{acmart}\settopmatter{printfolios=true}
%% For final camera-ready submission, w/ required CCS and ACM Reference
%\documentclass[sigplan]{acmart}\settopmatter{}

%% Conference information
%% Supplied to authors by publisher for camera-ready submission;
%% use defaults for review submission.
\acmConference[POPL 2018]{}{January 07--13, 2018}{Los Angeles, CA, USA}
%\acmYear{2017}
%\acmISBN{} % \acmISBN{978-x-xxxx-xxxx-x/YY/MM}
%\acmDOI{} % \acmDOI{10.1145/nnnnnnn.nnnnnnn}
%\startPage{1}

%% Copyright information
%% Supplied to authors (based on authors' rights management selection;
%% see authors.acm.org) by publisher for camera-ready submission;
%% use 'none' for review submission.
\setcopyright{none}
%\setcopyright{acmcopyright}
%\setcopyright{acmlicensed}
%\setcopyright{rightsretained}
%\copyrightyear{2017}           %% If different from \acmYear

%% Bibliography style
\bibliographystyle{acm-reference-format}
%% Citation style
\citestyle{acmnumeric}     %% For numeric citations


%%%%%%%%%%%%%%%%%%%%%%%%%%%%%%%%%%%%%%%%%%%%%%%%%%%%%%%%%%%%%%%%%%%%%%
%% Note: Authors migrating a paper from traditional SIGPLAN
%% proceedings format to PACMPL format must update the
%% '\documentclass' and topmatter commands above; see
%% 'acmart-pacmpl-template.tex'.
%%%%%%%%%%%%%%%%%%%%%%%%%%%%%%%%%%%%%%%%%%%%%%%%%%%%%%%%%%%%%%%%%%%%%%


%% Some recommended packages.
\usepackage{booktabs}   %% For formal tables:
                        %% http://ctan.org/pkg/booktabs
\usepackage{subcaption} %% For complex figures with subfigures/subcaptions
                        %% http://ctan.org/pkg/subcaption

\usepackage[T1]{fontenc}
\usepackage[utf8]{inputenc}
\usepackage{listings}
\usepackage{todonotes}

\newcommand{\ottifc}{Ott-IFC}
\newcommand{\config}[3]{\ensuremath{\langle #1, #2, #3 \rangle}}
\newcommand{\emits}[3]{\ensuremath{#2\downarrow_{#1}#3}}
\newcommand{\proj}[2]{\ensuremath{{#1 \! \upharpoonright  \! {#2}}}}
\newcommand{\step}{{\ensuremath{\Downarrow}}}

\definecolor{grey}{rgb}{0.8,0.8,0.8}
\definecolor{code-background}{RGB}{255, 248, 220}
\definecolor{code-comment}{RGB}{196, 42, 42}
\definecolor{code-linenumber}{rgb}{0.5,0.5,0.5}
\definecolor{code-keyword}{RGB}{148, 0, 211}
\definecolor{coqatoo-pink}      {RGB}{255, 107, 104}
\definecolor{coqatoo-gray}      {RGB}{64, 64, 64}
\definecolor{coqatoo-yellow}    {RGB}{239, 223, 0}
\lstset{
   backgroundcolor=\color{code-background},   	% choose the background color; you must add \usepackage{color} or \usepackage{xcolor}
   basicstyle=\tt\footnotesize,       			% the size of the fonts that are used for the code
   breakatwhitespace=false,         			% sets if automatic breaks should only happen at whitespace
   breaklines=true,                 			% sets automatic line breaking
   captionpos=none,                    			% sets the caption-position to bottom
   commentstyle=\color{code-comment},   		% comment style
   deletekeywords={...},            			% if you want to delete keywords from the given language
   escapeinside={\%*}{*)},          			% if you want to add LaTeX within your code
   extendedchars=true,              			% lets you use non-ASCII characters; for 8-bits encodings only, does not work with UTF-8
   frame=tb,
   framerule=0pt,
   framextopmargin=0pt,
   framexbottommargin=0pt,
   keepspaces=true,                 			% keeps spaces in text, useful for keeping indentation of code (possibly needs columns=flexible)
   keywordstyle=\color{code-keyword},       						% keyword style
   language=ML,                 				% the language of the code
   morekeywords={Lemma, Proof, Qed, if, then, else, end, skip, stop, read, from, write, to},            % if you want to add more keywords to the set
   numbers=none,                    			% where to put the line-numbers; possible values are (none, left, right)
   numbersep=5pt,	                   			% how far the line-numbers are from the code
   numberstyle=\tiny\color{code-linenumber}, 	% the style that is used for the line-numbers
   rulecolor=\color{black}, 	        		% if not set, the frame-color may be changed on line-breaks within not-black text (e.g. comments (green here))
   showspaces=false,            	    		% show spaces everywhere adding particular underscores; it overrides 'showstringspaces'
   showstringspaces=false,          			% underline spaces within strings only
   showtabs=false,                  			% show tabs within strings adding particular underscores
   stepnumber=1,                    			% the step between two line-numbers. If it's 1, each line will be numbered
   stringstyle=\color{black},            		% string literal style
   tabsize=2,                       			% sets default tabsize to 2 spaces
   gobble=0,									% number of characters to remove at the beginning of each line
   mathescape=true,								% to render math symbols in the listing (between $)
   title=\lstname,                   			% show the filename of files included with \lstinputlisting; also try caption instead of title
   belowcaptionskip = 0cm
}

\begin{document}

%% Title information
\title[ott-ifc]{Generating Information-Flow Control Mechanisms from Programming Language Specifications}


%% Author information
\author{Andrew Bedford}
\orcid{0000-0003-3101-4272}             %% \orcid is optional
\affiliation{
  \institution{Laval University}            %% \institution is required
  \state{Quebec}
  \country{Canada}                    %% \country is recommended
}
\email{andrew.bedford.1@ulaval.ca}          %% \email is recommended


%% Abstract
%\begin{abstract}
%We present \ottifc\, a tool that can automatically generate information-flow control mechanisms from programming language specifications (i.e., syntax and semantics).
%\end{abstract}

\maketitle

%Problem and Motivation: Clearly state the problem being addressed and explain the reasons for seeking a solution to this problem.
\section{Problem and Motivation}
Modern operating systems rely mostly on access-control mechanisms to protect users information. However, access control mechanisms are insufficient as they cannot regulate the propagation of information once it has been released for processing. To address this issue, a new research trend called \emph{language-based information-flow security}~\cite{DBLP:journals/jsac/SabelfeldM03} has emerged. The idea is to use techniques from programming languages, such as program analysis, monitoring, rewriting and type checking, to enforce information-flow policies (e.g., information from a private file should not saved in a public file). Mechanisms that enforce such policies (e.g., \cite{DBLP:journals/jcs/VolpanoIS96, DBLP:conf/csfw/ChudnovN10, DBLP:conf/csfw/AskarovCM15, DBLP:journals/compsec/BedfordCDKT17}) are called \emph{information-flow control mechanisms}. 

Developping sound information-flow control mechanisms can be a laborious and error-prone task, particularly when dealing with complex programming languages, due to the numerous ways through which information may flow in a program.

%Background and Related Work: Describe the specialized (but pertinent) background necessary to appreciate the work. Include references to the literature where appropriate, and briefly explain where your work departs from that done by others.
\section{Background and Related Work}\label{section:background}
Most information-flow control mechanisms seek to enforce a policy called \emph{non-interference}~\cite{DBLP:conf/sp/GoguenM82a}, which essentially states that private information may not interfere with the publicly observable behavior of a program. To enforce non-interference, a mechanism must take into account two types of information flows: \emph{explicit flows} and \emph{implicit flows}~\cite{DBLP:journals/cacm/Denning76}. 

An insecure explicit information flow (Listing~\ref{listing:explicit-flow}) occurs when private information flows directly into public information. 
\begin{lstlisting}[captionpos=b, caption=Insecure explicit flow, label=listing:explicit-flow]
  public := private
\end{lstlisting}
Explicit flows can be prevented by associating labels to sensitive information and propagating them whenever the information is used; a process known as \emph{tainting}.

An insecure implicit information flow (Listing~\ref{listing:implicit-flow}) occurs when private information influences public information\\through the control-flow of the application.
\pagebreak
\begin{lstlisting}[captionpos=b, caption=Insecure implicit flow, label=listing:implicit-flow]
  if (private > 0) then
    public := 0
  else
    public := 1
  end
\end{lstlisting}
Implicit flows can be prevented using a program counter, $pc$, which keeps track of the context in which a command is executed.

%Approach and Uniqueness: Describe your approach in attacking the problem and clearly state how your approach is novel.
\section{Approach and Uniqueness}
Information-flow control mechanisms are usually designed and implemented from the ground-up by a human. In order to make this process less laborious and reduce the risk of errors, we have created a tool called \ottifc\ that takes as input a programming language's specification (i.e., syntax and semantics) and produces a mechanism's specification (e.g., instrumented semantics). 

As the name implies, the specifications that \ottifc\ takes as input (and outputs) are written in Ott~\cite{DBLP:journals/jfp/SewellNOPRSS10}. Ott is a tool that can generate LaTeX, Coq or Isabelle/HOL versions of a programming language's specification. The specification is written in a concise and readable ASCII notation that ressembles what one would write in informal mathematics (see following Listings).

Hence, the development process of a mechanism using Ott and \ottifc\ would look like this:
\begin{enumerate}
  \item Write a specification of the language on which we want to enforce non-interference in Ott.
  \item Use \ottifc\ to generate the mechanism.
  \item Use Ott to export the mechanism to LaTeX/Coq/Isabelle/HOL and complete the implementation.
\end{enumerate}

For the moment, \ottifc\ supports only languages whose specification respects certain rules. Namely, that the syntax be composed of expressions, which may only read the memory, and commands, which may read or write the memory; and that the program configurations be of the form $\langle command, memory\rangle$. 

To illustrate our approach, consider the imperative language whose syntax is defined in Listing~\ref{listing:input-syntax} and (partial) semantics in Listings~\ref{listing:input-semantics-assign} and \ref{listing:input-semantics-if}.
\newpage
\begin{lstlisting}[label=listing:input-syntax,captionpos=b,caption=Ott syntax of a simple imperative language]
  arith_expr, a ::= x | n | a1 + a2 | a1 * a2 
  bool_expr, b ::= true | false | a1 < a2
  commands, c ::= skip | x := a | c1 ; c2 | 
                  if b then c1 else c2 end | 
                  while b do c end   
\end{lstlisting}

\begin{lstlisting}[label=listing:input-semantics-assign,captionpos=b,caption={Ott big-step semantics of the assign command}]
  <a, m> || <n, m>
  ---------------------------------
  <x := a, m> || <skip, m[x |-> n]>
\end{lstlisting}

\begin{lstlisting}[label=listing:input-semantics-if, captionpos=b,caption={Ott big-step semantics of the if command}]
  <b, m> || <true, m>
  <c1, m> || <skip, m1>
  --------------------------------------------
  <if b then c1 else c2 end, m> || <skip, m1>

  <b, m> || <false, m>
  <c2, m> || <skip, m2>
  --------------------------------------------
  <if b then c1 else c2 end, m> || <skip, m2>
\end{lstlisting}

In order to automatically apply the techniques described in Section~\ref{section:background}, \ottifc\ starts by inserting a typing environment \lstinline{E} (which maps variables to their label) and a program counter \lstinline{pc} in each semantic rules. That is, the configurations \lstinline{<c, m>} are changed to \lstinline{<c, m, E, pc>}.

To prevent explicit flows, it identifies the semantic rules that may modify the memory \lstinline{m} (e.g., Listing~\ref{listing:input-semantics-assign}). In each of those rules, it inserts a condition to ensure that private information is not stored into a public variable and updates the modified variable's (\lstinline{x} here) label with the label of the expressions that are used in the rule.

%\begin{lstlisting}[label=listing:input-semantics-assign,captionpos=b,caption={Ott big-step semantics of the assign command}]
%  <a, m> || <n, m>
%  E |- a : ta
%  E |- x : tx
%  ta <= tx
%  ---------------------------------
%  <x:=a, m, E, pc> || <skip, m[x |-> n], E[x |-> ta], pc>
%\end{lstlisting}

To prevent implicit flows, it identifies commands that may influence the control-flow of the application. That is, commands for which a program configuration may lead to two different program configurations (e.g., Listing~\ref{listing:input-semantics-if}). It then updates to the program counter \lstinline{pc} with the level of the expressions that are present in the rule (only \lstinline{b} in this case).
%\begin{lstlisting}[label=listing:output-semantics-if, captionpos=b,caption={Instrumented semantics of the if command}]
%  <b, m> || <true, m>
%  E |- b : tb
%  <c1, m, E, pc |_| tb> || <skip, m1, E1, pc1>
%  --------------------------------------------
%  <if b then c1 else c2 end, m, E, pc> || <skip, m1, E1, pc>
% 
%  <b, m> || <false, m>
%  E |- b : tb
%  <c2, m, E, pc |_| tb> || <skip, m2, E2, pc2>
%  --------------------------------------------
%  <if b then c1 else c2 end, m, E, pc> || <skip, m2, E2, pc>
%\end{lstlisting}

%Results and Contributions: Clearly show how the results of your work contribute to computer science and explain the significance of those results.
\section{Current Status and Future Work}
We have implemented a prototype of our algorithm and validated that it works on two  imperative languages: one defined using small-step semantics and the other using big-step semantics. We have also begun to draft a soundness proof, that is, a proof showing that the generated mechanisms enforce non-interference.

Before our tool can be of real use to most researchers, much work remains to be done.

\paragraph{Language Support} The restrictions on the syntax and configurations means that only certain types of languages can be used in \ottifc. For example, most functional languages would not be supported because, in those languages, commands can be expressions. We are currently in the process of building a repository of formalized languages so that we can test and extend our approach to a wider range of languages.

\paragraph{Parametrization} For the moment, \ottifc\ only generates one type of information-flow control mechanism. We plan on parametrizing our tool so that users can choose the type of mechanism to generate (e.g., type system, monitor) and choose some of its features (e.g., flow-sensitivity, termination-sensitivity, progress-sensitivity).

\paragraph{Generating Formal Proofs} We expect that some users will use the mechanisms generated by \ottifc\ as a foundation to build better and more precise mechanisms. One of the most grueling task when building an information-flow control mechanism is to prove its soundness. In order to help those users, we plan on generating a skeleton of the proof in Coq or Isabelle/HOL (both languages are supported by Ott).

\paragraph{Verifying Existing Mechanisms} The same rules that \ottifc\ uses to generate sound mechanisms could be used to verify the soundness of existing mechanisms and identify potential errors.

%% Acknowledgments
\begin{acks}
We would like to thank Josée Desharnais and Nadia Tawbi for their support and the anonymous reviewers for their comments.
\end{acks}

%% Bibliography
\bibliography{references}

\end{document}
