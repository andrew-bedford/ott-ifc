%% For double-blind review submission, w/o CCS and ACM Reference (max submission space)
\documentclass[sigplan,10pt]{acmart}\settopmatter{printfolios=true,printccs=false,printacmref=false}
  
%% For double-blind review submission, w/ CCS and ACM Reference
%\documentclass[sigplan,review,anonymous]{acmart}\settopmatter{printfolios=true}
%% For single-blind review submission, w/o CCS and ACM Reference (max submission space)
%\documentclass[sigplan,review]{acmart}\settopmatter{printfolios=true,printccs=false,printacmref=false}
%% For single-blind review submission, w/ CCS and ACM Reference
%\documentclass[sigplan,review]{acmart}\settopmatter{printfolios=true}
%% For final camera-ready submission, w/ required CCS and ACM Reference
%\documentclass[sigplan]{acmart}\settopmatter{}

%% Conference information
%% Supplied to authors by publisher for camera-ready submission;
%% use defaults for review submission.
\acmConference[POPL 2018]{}{January 07--13, 2018}{Los Angeles, CA, USA}
%\acmYear{2017}
%\acmISBN{} % \acmISBN{978-x-xxxx-xxxx-x/YY/MM}
%\acmDOI{} % \acmDOI{10.1145/nnnnnnn.nnnnnnn}
%\startPage{1}

%% Copyright information
%% Supplied to authors (based on authors' rights management selection;
%% see authors.acm.org) by publisher for camera-ready submission;
%% use 'none' for review submission.
\setcopyright{none}
%\setcopyright{acmcopyright}
%\setcopyright{acmlicensed}
%\setcopyright{rightsretained}
%\copyrightyear{2017}           %% If different from \acmYear

%% Bibliography style
\bibliographystyle{acm-reference-format}
%% Citation style
\citestyle{acmnumeric}     %% For numeric citations


%%%%%%%%%%%%%%%%%%%%%%%%%%%%%%%%%%%%%%%%%%%%%%%%%%%%%%%%%%%%%%%%%%%%%%
%% Note: Authors migrating a paper from traditional SIGPLAN
%% proceedings format to PACMPL format must update the
%% '\documentclass' and topmatter commands above; see
%% 'acmart-pacmpl-template.tex'.
%%%%%%%%%%%%%%%%%%%%%%%%%%%%%%%%%%%%%%%%%%%%%%%%%%%%%%%%%%%%%%%%%%%%%%


%% Some recommended packages.
\usepackage{booktabs}   %% For formal tables:
                        %% http://ctan.org/pkg/booktabs
\usepackage{subcaption} %% For complex figures with subfigures/subcaptions
                        %% http://ctan.org/pkg/subcaption

\usepackage[T1]{fontenc}
\usepackage[utf8]{inputenc}
\usepackage{listings}

\newcommand{\ottifc}{Ott-IFC}
\newcommand{\config}[3]{\ensuremath{\langle #1, #2, #3 \rangle}}
\newcommand{\emits}[3]{\ensuremath{#2\downarrow_{#1}#3}}
\newcommand{\proj}[2]{\ensuremath{{#1 \! \upharpoonright  \! {#2}}}}
\newcommand{\step}{{\ensuremath{\Downarrow}}}

\definecolor{grey}{rgb}{0.8,0.8,0.8}
\definecolor{code-background}{RGB}{255, 248, 220}
\definecolor{code-comment}{RGB}{196, 42, 42}
\definecolor{code-linenumber}{rgb}{0.5,0.5,0.5}
\definecolor{code-keyword}{RGB}{148, 0, 211}
\definecolor{coqatoo-pink}      {RGB}{255, 107, 104}
\definecolor{coqatoo-gray}      {RGB}{64, 64, 64}
\definecolor{coqatoo-yellow}    {RGB}{239, 223, 0}
\lstset{
   backgroundcolor=\color{code-background},   	% choose the background color; you must add \usepackage{color} or \usepackage{xcolor}
   basicstyle=\tt\footnotesize,       			% the size of the fonts that are used for the code
   breakatwhitespace=false,         			% sets if automatic breaks should only happen at whitespace
   breaklines=true,                 			% sets automatic line breaking
   captionpos=none,                    			% sets the caption-position to bottom
   commentstyle=\color{code-comment},   		% comment style
   deletekeywords={...},            			% if you want to delete keywords from the given language
   escapeinside={\%*}{*)},          			% if you want to add LaTeX within your code
   extendedchars=true,              			% lets you use non-ASCII characters; for 8-bits encodings only, does not work with UTF-8
   frame=tb,
   framerule=0pt,
   framextopmargin=0pt,
   framexbottommargin=0pt,
   keepspaces=true,                 			% keeps spaces in text, useful for keeping indentation of code (possibly needs columns=flexible)
   keywordstyle=\color{code-keyword},       						% keyword style
   language=ML,                 				% the language of the code
   morekeywords={Lemma, Proof, Qed, if, then, else, end, skip, stop, read, from, write, to},            % if you want to add more keywords to the set
   numbers=none,                    			% where to put the line-numbers; possible values are (none, left, right)
   numbersep=5pt,	                   			% how far the line-numbers are from the code
   numberstyle=\tiny\color{code-linenumber}, 	% the style that is used for the line-numbers
   rulecolor=\color{black}, 	        		% if not set, the frame-color may be changed on line-breaks within not-black text (e.g. comments (green here))
   showspaces=false,            	    		% show spaces everywhere adding particular underscores; it overrides 'showstringspaces'
   showstringspaces=false,          			% underline spaces within strings only
   showtabs=false,                  			% show tabs within strings adding particular underscores
   stepnumber=1,                    			% the step between two line-numbers. If it's 1, each line will be numbered
   stringstyle=\color{black},            		% string literal style
   tabsize=2,                       			% sets default tabsize to 2 spaces
   gobble=0,									% number of characters to remove at the beginning of each line
   mathescape=true,								% to render math symbols in the listing (between $)
   title=\lstname,                   			% show the filename of files included with \lstinputlisting; also try caption instead of title
   belowcaptionskip = 0cm
}

\begin{document}

%% Title information
\title[ott-ifc]{Generating Information-Flow Control Mechanisms from Programming Language Specifications}


%% Author information
\author{Andrew Bedford}
\orcid{0000-0003-3101-4272}             %% \orcid is optional
\affiliation{
  \institution{Université Laval}            %% \institution is required
  \state{Quebec}
  \country{Canada}                    %% \country is recommended
}
\email{andrew.bedford.1@ulaval.ca}          %% \email is recommended


%% Abstract
\begin{abstract}
We present a tool called \ottifc, that can automatically generate information-flow control mechanisms from programming language specifications (i.e., syntax and semantics).
\end{abstract}

\maketitle

%Problem and Motivation: Clearly state the problem being addressed and explain the reasons for seeking a solution to this problem.
\section{Introduction}
Language-based security~\cite{DBLP:conf/dagstuhl/SchneiderMH01} is an active field of research in which techniques from programming languages, such as program analysis, monitoring, rewriting and type checking, are used to enforce security policies. One of the most studied security policy is called \emph{non-interference}. It essentially states that private information may not interfere with the publicly observable behavior of a program. For example, Listings~\ref{listing:explicit-flow} and \ref{listing:implicit-flow} both violate non-interference because someone observing the contents of \lstinline{public} could learn something about \lstinline{private}.

\begin{lstlisting}[captionpos=b, caption=Insecure explicit flow, label=listing:explicit-flow]
  public := private
\end{lstlisting}

\begin{lstlisting}[captionpos=b, caption=Insecure implicit flow, label=listing:implicit-flow]
  if (private > 0) then
    public := 0
  else
    public := 1
  end
\end{lstlisting}

Mechanisms that enforce non-interference are called \\\emph{information-flow control mechanisms}, as they track and control where information may flow during the execution of a program.



In order to help language-based security researchers develop sound

%there are only a handful of information-flow control mechanisms for "real" languages. most papers

%Background and Related Work: Describe the specialized (but pertinent) background necessary to appreciate the work. Include references to the literature where appropriate, and briefly explain where your work departs from that done by others.


\section{Approach and Uniqueness}
We chose to use Ott~\cite{DBLP:journals/jfp/SewellNOPRSS10} as our input language. Ott is a programming language specification tool which 



\begin{lstlisting}[captionpos=b,caption=Syntax of a simple imperative language]
arith_expr, a :: ae_ ::=
  | x         
  | n                        
  | a1 + a2                  
  | a1 * a2                       

bool_expr, b :: be_ ::=
  | true                        
  | false   
  | a1 < a2          

commands, c :: cmd_ ::=
  | skip    
  | x := a 
  | c1 ; c2           
  | if b then c1 else c2 end
  | while b do c end   
\end{lstlisting}

\begin{lstlisting}
  %%% Assignment %%%
  <a, s> || <n, s>
  ---------------------------------
  <x := a, s> || <skip, s[x |-> n]>
\end{lstlisting}

\begin{lstlisting}
  %%% If %%%
  <b, s> || <true, s>
  <c1, s> || <skip, s'>
  --------------------------------------------
  <if b then c1 else c2 end, s> || <skip, s'>
  
  <b, s> || <false, s>
  <c2, s> || <skip, s'>
  --------------------------------------------
  <if b then c1 else c2 end, s> || <skip, s'>
\end{lstlisting}

\section{Current Status and Future Work}
We have implemented a prototype of our algorithm and validated that it works on two simple imperative languages: one defined using small-step semantics and the other using large-step semantics. We have also begun to draft a soundness proof, that is, a proof showing that the generated mechanisms enforce non-interference.

Before our tool can be of real use to most researchers, much work remains to be done.

\paragraph{Language Support} \ottifc~ currently makes two assumptions about the language: (1) that the syntax be composed of expressions, which may only read the memory, and commands, which may read or write the memory; and (2) that the program configurations be of the form $\langle command, memory\rangle$. While these assumptions helped simplify the implementation, it also restricts the types of languages that can be used in \ottifc. For example, most functional languages would not be supported because in those languages functions can be expressions. 

\paragraph{Language Repository} We are currently in the process of building a repository of languages so that we can test our approach on additional languages. 

\paragraph{Parametrization} For the moment, \ottifc~ only generates one type of information-flow control mechanism: a type system which enforces termination-insensitive non-interference. We plan on parametrizing our tool so that users can choose the type of mechanism to generate (e.g., type system, monitor) and choose some of its features (e.g., flow-sensitivity, termination-sensitivity, progress-sensitivity).

\paragraph{Generating Formal Proofs} We expect that some users will use the mechanisms generated by \ottifc~ as a foundation to build better and more precise mechanisms. One of the most grueling task when building an information-flow control mechanism is to prove its soundness. In order to help those users, we plan on generating a skeleton of the proof in Coq or Isabelle/HOL (both languages are supported by Ott).

%% Acknowledgments
\begin{acks}
We would like to thank Josée Desharnais and Nadia Tawbi for their support and the anonymous reviewers for their comments.
\end{acks}

%% Bibliography
\bibliography{references}

\end{document}
