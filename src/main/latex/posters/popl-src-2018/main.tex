%see https://www.overleaf.com/10499096qftbbdzshvqy#/39157539/ or http://www.latextemplates.com/template/jacobs-landscape-posterfor example 

\documentclass[final]{beamer}
\usepackage[scale=1.1]{beamerposter} % Use the beamerposter package for laying out the poster
\usepackage[T1]{fontenc}
\usepackage[utf8]{inputenc}
\usepackage{listings}
\usepackage{mathpartir}
\usepackage{graphicx}
\newcommand{\ottifc}{Ott-IFC}
\newcommand{\config}[3]{\ensuremath{\langle #1, #2, #3 \rangle}}
\newcommand{\emits}[3]{\ensuremath{#2\downarrow_{#1}#3}}
\newcommand{\proj}[2]{\ensuremath{{#1 \! \upharpoonright  \! {#2}}}}
\newcommand{\step}{{\ensuremath{\Downarrow}}}
\newcommand{\levelOfChan}[1]{\ensuremath {\mathit{labelOfChan}(#1)}}

\newtheorem{definition}{Definition}
\newtheorem{theorem}{Theorem}

%the 3 following lines makes the text in \lstinline the right size. Without it, the text is too small
\makeatletter
\lst@AddToHook{TextStyle}{\let\lst@basicstyle\ttfamily\normalsize\fontfamily{pcr}\selectfont}
\makeatother

\definecolor{grey}{rgb}{0.8,0.8,0.8}
\definecolor{code-background}{RGB}{255, 248, 220}
\definecolor{code-comment}{RGB}{196, 42, 42}
\definecolor{code-linenumber}{rgb}{0.5,0.5,0.5}
\definecolor{code-keyword}{RGB}{148, 0, 211}
\definecolor{coqatoo-pink}      {RGB}{255, 107, 104}
\definecolor{coqatoo-gray}      {RGB}{64, 64, 64}
\definecolor{coqatoo-yellow}    {RGB}{239, 223, 0}
\lstset{
   backgroundcolor=\color{code-background},   	% choose the background color; you must add \usepackage{color} or \usepackage{xcolor}
   basicstyle=\tt\footnotesize,       			% the size of the fonts that are used for the code
   breakatwhitespace=false,         			% sets if automatic breaks should only happen at whitespace
   breaklines=true,                 			% sets automatic line breaking
   captionpos=none,                    			% sets the caption-position to bottom
   commentstyle=\color{code-comment},   		% comment style
   deletekeywords={...},            			% if you want to delete keywords from the given language
%   escapeinside={\%*}{*)},          			% if you want to add LaTeX within your code
   extendedchars=true,              			% lets you use non-ASCII characters; for 8-bits encodings only, does not work with UTF-8
   frame=tb,
   framerule=0pt,
   framextopmargin=0pt,
   framexbottommargin=0pt,
   keepspaces=true,                 			% keeps spaces in text, useful for keeping indentation of code (possibly needs columns=flexible)
   keywordstyle=\color{code-keyword},       						% keyword style
   language=ML,                 				% the language of the code
   morekeywords={Lemma, Proof, Qed, if, then, else, end, skip, stop, read, from, write, to},            % if you want to add more keywords to the set
   morecomment=[l]{\%},
   numbers=none,                    			% where to put the line-numbers; possible values are (none, left, right)
   numbersep=5pt,	                   			% how far the line-numbers are from the code
   numberstyle=\tiny\color{code-linenumber}, 	% the style that is used for the line-numbers
   rulecolor=\color{black}, 	        		% if not set, the frame-color may be changed on line-breaks within not-black text (e.g. comments (green here))
   showspaces=false,            	    		% show spaces everywhere adding particular underscores; it overrides 'showstringspaces'
   showstringspaces=false,          			% underline spaces within strings only
   showtabs=false,                  			% show tabs within strings adding particular underscores
   stepnumber=1,                    			% the step between two line-numbers. If it's 1, each line will be numbered
   stringstyle=\color{black},            		% string literal style
   tabsize=1,                       			% sets default tabsize to 2 spaces
   gobble=0,									% number of characters to remove at the beginning of each line
   mathescape=true,								% to render math symbols in the listing (between $)
   title=\lstname,                   			% show the filename of files included with \lstinputlisting; also try caption instead of title
   belowcaptionskip = 0cm
}
\usepackage{booktabs} % Top and bottom rules for tables
\usepackage{mathpartir}

\usetheme{confposter} % Use the confposter theme supplied with this template

\setbeamercolor{title in headline}{fg=gray!125}
\setbeamercolor{block title}{fg=gray!125,bg=white} % Colors of the block titles
\setbeamercolor{block body}{fg=black,bg=white} % Colors of the body of blocks
\setbeamercolor{block alerted title}{fg=white,bg=dblue!70} % Colors of the highlighted block titles
\setbeamercolor{block alerted body}{fg=black,bg=dblue!10} % Colors of the body of highlighted blocks
% Many more colors are available for use in beamerthemeconfposter.sty

%-----------------------------------------------------------
% Define the column widths and overall poster size
% To set effective sepwid, onecolwid and twocolwid values, first choose how many columns you want and how much separation you want between columns
% In this template, the separation width chosen is 0.024 of the paper width and a 4-column layout
% onecolwid should therefore be (1-(# of columns+1)*sepwid)/# of columns e.g. (1-(4+1)*0.024)/4 = 0.22
% Set twocolwid to be (2*onecolwid)+sepwid = 0.464
% Set threecolwid to be (3*onecolwid)+2*sepwid = 0.708

\newlength{\sepwid}
\newlength{\onecolwid}
\newlength{\twocolwid}
\newlength{\threecolwid}
\setlength{\paperwidth}{48in} % A0 width: 46.8in
\setlength{\paperheight}{36in} % A0 height: 33.1in
\setlength{\sepwid}{0.024\paperwidth} % Separation width (white space) between columns
\setlength{\onecolwid}{0.22\paperwidth} % Width of one column
\setlength{\twocolwid}{0.464\paperwidth} % Width of two columns
\setlength{\threecolwid}{0.708\paperwidth} % Width of three columns
\setlength{\topmargin}{-0.5in} % Reduce the top margin size
%-----------------------------------------------------------

%----------------------------------------------------------------------------------------
%	TITLE SECTION 
%----------------------------------------------------------------------------------------

\title{Generating Information-Flow Control Mechanisms from\\ Programming Language Specifications} % Poster title

\author{Andrew Bedford (andrew.bedford.1@ulaval.ca)} % Author(s)

\institute{Laval University, Canada} % Institution(s)

%----------------------------------------------------------------------------------------

\AtBeginDocument{\usebeamerfont{normal text}}
\DeclareTextFontCommand{\emph}{\bfseries}
\begin{document}

\addtobeamertemplate{block end}{}{\vspace*{2ex}} % White space under blocks
\addtobeamertemplate{block alerted end}{}{\vspace*{2ex}} % White space under highlighted (alert) blocks

\setlength{\belowcaptionskip}{2ex} % White space under figures
\setlength\belowdisplayshortskip{2ex} % White space under equations

\begin{frame}[fragile] % The whole poster is enclosed in one beamer frame

\begin{columns}[t] % The whole poster consists of three major columns, the second of which is split into two columns twice - the [t] option aligns each column's content to the top

\begin{column}{\sepwid}\end{column} % Empty spacer column

\begin{column}{\onecolwid} % The first column

%\begin{alertblock}{Abstract}
%We introduce the concept of \emph{fading labels}. Fading labels are security labels that stop propagating their taint after a fixed amount of uses. Their use allows mechanisms to spend more resources on more important information.
%\end{alertblock}

%----------------------------------------------------------------------------------------
%	INTRODUCTION
%----------------------------------------------------------------------------------------

\begin{block}{\textsc{Motivation}}
    Modern operating systems rely on access-control mechanisms to protect users information. However, these mechanisms are insufficient as they cannot regulate the propagation of information once it has been released. 
    
    \hspace{1.5cm}To address this issue, a new research trend called \emph{language-based information-flow security}~\cite{DBLP:journals/jsac/SabelfeldM03} has emerged. The idea is to use techniques from programming languages, such as program analysis and type checking, to enforce information-flow policies. Mechanisms that enforce such policies are called \emph{information-flow control mechanisms}. 
\end{block}

\setbeamercolor{block alerted title}{fg=white,bg=coqatoo-pink}
\begin{alertblock}{Problem}
    Developing sound information-flow control mechanisms can be a laborious and error-prone task due to the numerous ways through which information may flow in a program.
\end{alertblock}

\begin{block}{\textsc{Background}}
    Most information-flow control mechanisms seek to enforce a policy called \emph{non-interference}~\cite{DBLP:conf/sp/GoguenM82a}, which states that private information may not interfere with the publicly observable behavior of a program. More formally:
    \vspace{-0.31cm}
    \setbeamercolor{block alerted title}{bg=white}
    \begin{alertblock}{}
        \textit{A program $p$ satisfies non-interference if for any $\ell \in {\cal L}$, and for any two memories $m$ and $m'$ that are $\ell$-equivalent, and for any trace $o$ such that $\emits{}{\config{p}{m}{\epsilon}}{o} $, then there is some trace $o'$, such that $\emits{}{\config{p}{m'}{\epsilon}}{o'}$ and $\proj{o}{\ell}$ is a prefix of $\proj{o'}{\ell}$ (or vice versa).}
    \end{alertblock}
    \vspace{-1cm}
    \color{black}
    
    
    To enforce non-interference, two types of information flows must be taken into account: 
    \begin{enumerate}
        \item{\emph{Explicit flows} occur when private information flows directly into public information.
            \begin{lstlisting}[label=listing:explicit-flow,gobble=15]
                public := private
            \end{lstlisting}} 
        \item{\emph{Implicit flows} occur when private information influences public information through the control-flow of the application.
            \begin{lstlisting}[label=listing:implicit-flow,gobble=15]
                if (private > 0) then
                  public := 0
                else
                  public := 1
                end
            \end{lstlisting}}
    \end{enumerate}
\end{block}


\end{column}
\begin{column}{\twocolwid}

\begin{block}{\textsc{Approach and Uniqueness}}
These mechanisms are usually designed and implemented completely by a human. We have created a tool called \emph{\ottifc} that automates parts of the process. It takes as input a programming language's specification (i.e., syntax and semantics) and produces a mechanism's specification. 
%The development process of a mechanism using Ott and \ottifc\ looks like this:

%\noindent
%\begin{minipage}[t]{0.48\linewidth}
%\begin{enumerate}
%\item Write a specification of the language on which we want to enforce non-interference in Ott.
%\item Use \ottifc\ to generate the mechanism.
%\item Use Ott to export the mechanism to LaTeX/Coq/Isabelle/HOL and complete the implementation.
%\end{enumerate}    
%\end{minipage}
%\hfill%
%\begin{minipage}[t]{0.48\linewidth}
\begin{alertblock}{Example}
\begin{lstlisting}
arith_expr, a ::= x | n | a1 + a2 | a1 * a2                                    bool_expr, b ::= true | false | a1 < a2
commands, c ::= skip | x := a | c1 ; c2 | if b then c1 else c2 end | while b do c end | read x from ch | write x to ch  
\end{lstlisting}
\end{alertblock}      
%\end{minipage}
\end{block}
\vspace{-1.5cm}
\color{black}
To prevent explicit flows, Ott-IFC identifies the semantic rules that may modify the memory m (e.g., rule assign). In each of those rules, it updates the modified variable's label with the label of the expressions that are used in the rule.
\noindent
\begin{minipage}[t]{0.48\linewidth}
\begin{alertblock}{Input}
\begin{lstlisting}
$$
<a, m, o> || <n, m, o>
-----------------------------------------------------
<x := a, m, o> || <stop, m[x |-> n], o>
$$
\end{lstlisting}        
\end{alertblock}
\end{minipage}
\hfill%
\begin{minipage}[t]{0.48\linewidth}
\setbeamercolor{block alerted title}{fg=white,bg=norange}
\begin{alertblock}{Output}
\begin{lstlisting}
<a, m, o, pc, E> || <n, m, o, pc, E>
E |- a : la
-----------------------------------------------------
<x := a, m, o, pc, E> || <stop, m[x |-> n], o, pc, 
  E[x |-> pc |_| la]>
\end{lstlisting}
\end{alertblock}
\end{minipage}\\
\color{black}
If an output is produced, it inserts a guard condition to ensure that no leak occurs.
\noindent
\begin{minipage}[t]{0.48\linewidth}
\begin{alertblock}{Input}
\begin{lstlisting}
$$
$$
$$
m(x) = n
-----------------------------------------------------
<write x to ch, m, o> || <stop, m[ch |-> n], 
  o::(ch ; n)>
\end{lstlisting}        
\end{alertblock}
\end{minipage}
\hfill%
\begin{minipage}[t]{0.48\linewidth}
\setbeamercolor{block alerted title}{fg=white,bg=norange}
\begin{alertblock}{Output}
\begin{lstlisting}
E |- x : lx
E |- ch : lch
lx |_| pc <= lch
m(x) = n
-----------------------------------------------------
<write x to ch, m, o, pc, E> ||  <stop, m[ch |-> n], 
  o::(ch ; n), pc, E>
\end{lstlisting}
\end{alertblock}
\end{minipage}

\color{black}
To prevent implicit flows, it identifies commands that may influence the control-flow of the application. It then updates to the program counter pc with the level of the expressions that are present in the rule.\\
\noindent
\begin{minipage}[t]{0.48\linewidth}
\begin{alertblock}{Input}
\begin{lstlisting}
$$
$$
<b, m, o> || <true, m, o>
<c1, m, o> || <stop, m1, o1>
-----------------------------------------------------
<if b then c1 else c2 end, m, o> || <stop, m1, o1>




<b, m, o> || <false, m, o>
<c2, m, o> || <stop, m2, o2>
-----------------------------------------------------
<if b then c1 else c2 end, m, o> || <stop, m2, o2>
$$
\end{lstlisting}        
\end{alertblock}
\end{minipage}
\hfill%
\begin{minipage}[t]{0.48\linewidth}
\setbeamercolor{block alerted title}{fg=white,bg=norange}
\begin{alertblock}{Output}
\begin{lstlisting}
E |- b : lb
<b, m, o, pc, E> || <true, m, o, pc, E>
<c1, m, o, pc |_| lb, E> || <stop, m1, o1, pc |_| lb, E>
E1 = updateModifVars(c2)
-----------------------------------------------------
<if b then c1 else c2 end, m, o, pc, E> ||  <stop, m1, o1, pc, E1>

E |- b : lb
<b, m, o, pc, E> || <false, m, o, pc, E>
<c2, m, o, pc |_| lb, E> || <stop, m2, o2, pc |_| lb, E>
E2 = updateModifVars(c1)
-----------------------------------------------------
<if b then c1 else c2 end, m, o, pc, E> ||  <stop, m2, o2, pc, E2>
\end{lstlisting}
\end{alertblock}
\end{minipage}



%\setbeamercolor{block alerted title}{fg=white,bg=norange} % Change the alert block title colors
%\setbeamercolor{block alerted body}{fg=black,bg=white} % Change the alert block body colors




    %\begin{figure}[ht]
    %\includegraphics[width=\columnwidth]{images/derivations-graph-large.png}
    %\caption{PDG-like representation of Listing~\ref{listing:lost-information}}
    %\label{figure:derivations}
    %\end{figure}
%\end{alertblock}

\end{column}

\begin{column}{\onecolwid}

\begin{block}{\textsc{Current Status}}
We have implemented a prototype of our algorithm~\cite{GitHub:ott-ifc} and validated that it works on two imperative languages. It currently supports languages whose specification: 
\begin{enumerate}
\item is composed of expressions, which may only read the memory, and commands, which may read or write the memory
\item states are of the form $\langle command, memory, outputs\rangle$. 
\end{enumerate}

We have also begun to draft a soundness proof, that is, a proof showing that the generated mechanisms enforce non-interference.


\end{block}

\begin{block}{\textsc{Future Work}}
    \begin{itemize}
        \item Add support for a greater variety of languages
        \item Parametrize Ott-IFC so that it can generate multiple types of mechanisms
        \item Automatically generate a skeleton of proof in Coq
        \item Use Ott-IFC's rewriting rules to verify existing mechanisms
    \end{itemize}
\end{block}

\begin{block}{\textsc{Acknowledgements}}
    We would like to thank Josée Desharnais, Nadia Tawbi for their support and the anonymous reviewers for their comments.    
\end{block}

\begin{block}{References}
    \bibliographystyle{acm}
    \bibliography{references}
\end{block}
\end{column}
\end{columns}
\end{frame}

\end{document}

